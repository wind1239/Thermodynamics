
%%%
%%% CHAPTER
%%%
\chapter{Thermodynamic Formulation}\label{Chapter:ThermodynamicFormulation}
 
%%% SECTION
\section{Mass Balance}\label{Chapter:ThermodynamicFormulation:Section:MassBalance}
Given a closed system at constant pressure and temperature conditions with $n_{c}$ components contained in $n_{p}$ phases. We can define a normalised concentration, i.e., mole fraction $\left(\mfr[x]{i}{j}\right)$ based on the relationship between the number of moles of component $i$ at phase $j$ with the total number of moles contained in the same phase, 
\begin{equation}
    \mfr[x]{i}{j} = \frc{\mfr[n]{i}{j}}{\mfr[n]{}{j}} = \frc{\frc{\mfr[m]{i}{j}}{MW_{i}}}{\frc{\mfr[m]{}{j}}{\mfr[\overline{MW}]{}{j}}}, \hfill \text{ with } i=1, \cdots, n_{c} \text{ and } j=1, \cdots, n_{p}
\label{Chapter:ThermodynamicFormulation:Eqn:MoleFractionDef}
\end{equation}
where $MW$ is the molar mass, $n$ is the number of moles and $m$ is the mass. In this Chapter, superscripts and subscripts stand for phase and component identities, respectively.  For completeness, we can also define
\begin{eqnarray}
    z_{i} = \frc{n_{i}}{n} &=& \frc{\frc{m_{i}}{MW_{i}}}{\frc{m}{\overline{MW}}}\label{Chapter:ThermodynamicFormulation:Eqn:FeedFractionDef} \\
    \mfr[\Pi]{}{j} = \frc{\mfr[n]{}{j}}{n} &=& \frc{\frc{\mfr[m]{}{j}}{\mfr[\overline{MW}]{}{j}}}{\frc{m}{\overline{MW}}},\label{Chapter:ThermodynamicFormulation:Eqn:PhaseFractionDef} 
\end{eqnarray}
where $z_{i}$ is the overall feed mass fraction of component $i$, and $\Pi$ is the mole fraction of phase $j$. The relations above are subject to the following constraints:
\begin{equation}
    \summation_{i=1}^{n_{c}}\mfr[x]{i}{j} = 1 \;\;\;,\;\;\; \summation_{i=1}^{n_{c}}z_{i} = 1 \;\;\;,\;\;\; \summation_{j=1}^{n_{p}}\mfr[\Pi]{}{j} = 1. \label{Chapter:ThermodynamicFormulation:Eqn:FractionConstraints1}
\end{equation}
These linear constraints can be expressed as
\begin{equation}
    \mfr[x]{n_{c}}{j}= 1 - \summation_{i=1}^{n_{c}-1}\mfr[x]{i}{j} \;\;\;,\;\;\; z_{n_{c}}= 1 - \summation_{i=1}^{n_{c}-1}z_{i} \;\;\;,\;\;\; \mfr[\Pi]{}{n_{p}} = 1 - \summation_{j=1}^{n_{p}-1}\mfr[\Pi]{}{j} \label{Chapter:ThermodynamicFormulation:Eqn:FractionConstraints2}
\end{equation}
Equation~\ref{Chapter:ThermodynamicFormulation:Eqn:FeedFractionDef} can be rearranged as,
\begin{eqnarray}
  z_{i} &=& \frc{n_{i}}{n} = \frc{\summation_{j=1}^{n_{p}}\mfr[n]{i}{j}}{n} \cdot \frc{\summation_{j=1}^{n_{p}}\mfr[n]{}{j}}{\summation_{j=1}^{n_{p}}\mfr[n]{}{j}} = \underbrace{\frc{ \summation_{j=1}^{n_{p}}\mfr[n]{}{j}}{n}}_{\red{\summation_{j=1}^{n_{p}}\mfr[\Pi]{}{j}}} \cdot \overbrace{ \frc{\summation_{j=1}^{n_{p}}\mfr[n]{i}{j}}{\summation_{j=1}^{n_{p}}\mfr[n]{}{j}}}^{\red{\summation_{j=1}^{n_{p}}\mfr[x]{i}{j}}}, \nonumber \\
       &=& \summation_{j=1}^{n_{p}} \mfr[\Pi]{}{j}\mfr[x]{i}{j},\;\;\;\;\;i=1, \cdots, n_{c}\hfill\label{Chapter:ThermodynamicFormulation:Eqn:FeedFractionDef2} 
\end{eqnarray}
with 
\begin{equation}
      0\leq \mfr[x]{i}{j} \leq 1 \;\;\;\text{ and }\;\;\; 0\leq\mfr[\Pi]{}{j}\leq 1 \hfill
\end{equation}
If the solution is contained in $n_{p}$ phases, the inequality can be rewritten as, 
\begin{equation}
    0 < \mfr[x]{i}{j} < 1 \;\;\;\text{ and }\;\;\;  0 < \mfr[\Pi]{}{j} < 1.
\end{equation}
And therefore
\begin{displaymath}
   \mfr[x]{p}{j} = 1 - \summation_{i=1,i\ne p}^{n_{c}}\mfr[x]{i}{j} \ne 0 \;\;\;\text{ and }\;\;\; \mfr[\Pi]{}{k} = 1 - \summation_{j=1,j\ne k}^{n_{p}}\mfr[\Pi]{}{j} \ne 0.
\end{displaymath}
Equation~\ref{Chapter:ThermodynamicFormulation:Eqn:FeedFractionDef2} may be rewritten as
\begin{equation}
   \mfr[x]{i}{k} = \frc{ z_{i} - \summation_{j=1,j\ne k}^{n_{p}} \mfr[\Pi]{}{j}\mfr[x]{i}{j}}{\mfr[\Pi]{}{k}} = \frc{ z_{i} - \summation_{j=1,j\ne k}^{n_{p}} \mfr[\Pi]{}{j}\mfr[x]{i}{j}}{ 1 - \summation_{j=1,j\ne k}^{n_{p}} \mfr[\Pi]{}{j}}\label{Chapter:ThermodynamicFormulation:Eqn:MoleFractionDef2}
\end{equation}

\begin{shaded}\noindent
   For 2 phases, $\Pi^{(1)}=L$ and $\Pi^{(2)}=V$, Eqn.~\ref{Chapter:ThermodynamicFormulation:Eqn:MoleFractionDef2} becomes:
     \begin{displaymath}
      \mfr[x]{i}{V} = \frc{z_{i}-L\mfr[x]{i}{L}}{1-L}, \hspace{1cm}\text{ for } i=1,2,\cdots,n_{c}
     \end{displaymath}
     And for 3 phases, $\Pi^{(1)}=L$, $\Pi^{(2)}=V$ and $\Pi^{(3)}=H$ (see Section~\ref{Chapter:Hydrate:Section:MassConservation}):
        \begin{displaymath}
           \mfr[x]{i}{H} = \frc{z_{i}-\left(V \mfr[x]{i}{V} + L \mfr[x]{i}{L}\right)}{1 - \left(V + L \right)}
        \end{displaymath}
\end{shaded}

%% SECTION
\section{Free Gibbs Energy}\label{Chapter:ThermodynamicFormulation:Section:GibbsEnergy}
By definition the chemical potential, $\mu$ is defined as
\begin{equation}
   \mu_{i} \equiv \Partial[G]{n_{i}}{T,P,n_{j\ne i}},
\end{equation}
where the Gibbs free energy, $G$, can be expressed as a function of the composition of the species present in the domain,
\begin{equation}
   G = G\left(n_{1}, n_{2}, \cdots, n_{n_{c}}\right).
\end{equation}
Now defining the molar Gibbs free energy, $g$,
\begin{displaymath}
   g = \frc{1}{n}G\left(n_{1}, n_{2}, \cdots, n_{n_{c}}\right).
\end{displaymath}
Assuming that $G$ is a {\it homogeneous function of the first order},
\begin{eqnarray}
   g &=& \frc{1}{n}G\left(n_{1}, n_{2}, \cdots, n_{n_{c}}\right) = G\left(\frc{n_{1}}{n}, \frc{n_{2}}{n}, \cdots, \frc{n_{n_{c}}}{n}\right) \nonumber \\
     &=& G\left(x_{1}, x_{2}, \cdots, x_{n_{c}}\right). \nonumber
\end{eqnarray}
This relation is {\it true only} on regions of {\it single phase}. In multiphase regions, the molar Gibbs free energy is given by,
\begin{equation}
   g = \mfr[g]{}{1} + \mfr[g]{}{2} + \cdots + \mfr[g]{}{n_{p}} = \summation_{j=1}^{n_{p}}\mfr[g]{}{j},
\end{equation}
with
\begin{displaymath}
   \mfr[g]{}{k} = \frc{1}{\mfr[n]{}{k}} \mfr[G]{}{k}\left(\mfr[n]{1}{k}, \mfr[n]{2}{k}, \cdots, \mfr[n]{n_{c}}{k}\right)
\end{displaymath}
As $\mfr[G]{}{k}$ is a homogeneous function of the first order,
\begin{eqnarray}
 \mfr[g]{}{k} &=& \frc{1}{\mfr[n]{}{k}} \mfr[G]{}{k}\left(\mfr[n]{1}{k}, \mfr[n]{2}{k}, \cdots, \mfr[n]{n_{c}}{k}\right) \nonumber \\
              &=&  \mfr[G]{}{k}\left( \frc{\mfr[n]{1}{k}}{\mfr[n]{}{k}}, \frc{\mfr[n]{2}{k}}{\mfr[n]{}{k}}, \cdots, \frc{\mfr[n]{n_{c}}{k}}{\mfr[n]{}{k}}\right) \nonumber \\
              &=&  \mfr[G]{}{k}\left(\mfr[x]{1}{k}, \mfr[x]{2}{k}, \cdots, \mfr[x]{n_{c}}{k}\right) \nonumber \\
    \text{ thus, }     && \nonumber\\
\mfr[g]{}{k} &=&  \mfr[g]{}{k}\left(\mfr[x]{1}{k}, \mfr[x]{2}{k}, \cdots, \mfr[x]{n_{c}}{k}\right). 
\end{eqnarray}

For two phase (liquid and vapour problems),
\begin{displaymath}
  \left \{
  \begin{aligned}
    & \mfr[g]{}{L} = \mfr[g]{}{L}\left(\mfr[x]{i}{L}\right) && \text{ with } \;\;\;i = 1, 2, \cdots, n_{c} \\
    & 0 \leq \mfr[x]{i}{L} \leq 1 
  \end{aligned} \right.
\end{displaymath} 

\begin{displaymath}
  \left \{
  \begin{aligned}
    & \mfr[g]{}{V} = \mfr[g]{}{V}\left(\mfr[x]{i}{V}\right) && \text{ with } \;\;\;i = 1, 2, \cdots, n_{c} \\
    & 0 \leq \mfr[x]{i}{V} \leq 1 
  \end{aligned} \right.
\end{displaymath} 

Therefore,
\begin{equation}
  \left \{
  \begin{aligned}
    & g = \mfr[g]{}{V}\left(\mfr[x]{i}{V}\right) + \mfr[g]{}{L}\left(\mfr[x]{i}{L}\right) && \text{ with } \;\;\;i = 1, 2, \cdots, n_{c} \\
    & 0 \leq \mfr[x]{i}{V} \leq 1 \text{ and } 0 \leq \mfr[x]{i}{L} \leq 1 
  \end{aligned} \right.
\end{equation} 

We can use the property of homogeneity of first order to describe a thermodynamic potential as (for $\forall\lambda > 0$),
\begin{displaymath}
   U\left(\lambda\cdot S, \lambda\cdot V,  \lambda\cdot n_{1}, \cdots, \lambda\cdot n_{n_{c}}\right) = \lambda U\left(S, V, n_{1}, \cdots, n_{n_{c}}\right),
\end{displaymath}
where $U$, $S$ and $V$ are internal energy, entropy and volume, respectively

If we differentiate both sides with respect to $\lambda$ and using the {\it chain rule} in the left-hand side,
\begin{eqnarray}
  &&\frc{\partial}{\partial S} U\left(\lambda\cdot S, \lambda\cdot V,  \lambda\cdot n_{1}, \cdots, \lambda\cdot n_{n_{c}}\right)S + \frc{\partial}{\partial V} U\left(\lambda\cdot S, \lambda\cdot V,  \lambda\cdot n_{1}, \cdots, \lambda\cdot n_{n_{c}}\right)V + \nonumber \\
 && \hfill \summation_{i=1}^{n_{c}} \frc{\partial}{\partial n_{i}} U\left(\lambda\cdot S, \lambda\cdot V,  \lambda\cdot n_{1}, \cdots, \lambda\cdot n_{n_{c}}\right)n_{i} = U\left(S, V, n_{1}, \cdots, n_{n_{c}}\right) \nonumber 
\end{eqnarray}
Assuming $\lambda=1$,
\begin{displaymath}
   \Partial[U]{S}{V,n_{i}}S + \Partial[U]{V}{S,n_{i}}V + \summation_{i=1}^{n_{c}}\Partial[U]{n_{i}}{V,S,n_{j\ne i}}n_{i} = U,
\end{displaymath}
and using the Maxwell relations to replace the differential quantities
\begin{equation}
   U = TS - PV + \summation_{i=1}^{n_{c}} \mu_{i}n_{i}\label{Chapter:ThermodynamicFormulation:Eqn:EulerRelation}
\end{equation}
This equation is known as {\it Euler relation}.  From the Gibbs free energy definition,
\begin{displaymath}
   G = U-TS+PV
\end{displaymath}
Replacing Euler relation in the equation above,
\begin{equation}
   G = \summation_{i=1}^{n_{c}} \mu_{i}n_{i},
\end{equation}
or for system with multiple phases in equilibrium,
\begin{equation}
   \mfr[G]{}{j} = \summation_{i=1}^{n_{c}}\mfr[\mu]{i}{j}\mfr[x]{i}{j},\;\;\;\;\; j=1, \cdots, n_{p}.
\end{equation}

For $T$ and $P$ constant, the chemical potential, $\mu$ can be written as a non-linear function
\begin{equation}
   \mfr[\mu]{i}{j} = \mfr[\mu]{i}{j}\left(\mfr[x]{1}{j} \cdots, \mfr[x]{n_{c}}{j}\right).\label{Chapter:ThermodynamicFormulation:Eqn:ChemPotDef2}
\end{equation}
The total Gibbs free energy can be then rewritten as
\begin{equation}
   G = \summation_{j=1}^{n_{p}} \mfr[G]{}{j} = \summation_{j=1}^{n_{p}}\summation_{i=1}^{n_{c}} \mfr[\mu]{i}{j}\mfr[n]{i}{j}.\label{Chapter:ThermodynamicFormulation:Eqn:GibbsTotal}
 \end{equation}

Assuming that we are dealing with closed system(i.e., $n$ is constant) and replacing Eqn.~\ref{Chapter:ThermodynamicFormulation:Eqn:GibbsTotal} in,
\begin{eqnarray}
     g = \frc{G}{n} &=& \frc{\summation_{i=1}^{n_{c}}\mfr[\mu]{i}{1}\mfr[n]{i}{1}}{n} +  \cdots +  \frc{\summation_{i=1}^{n_{c}}\mfr[\mu]{i}{n_{p}}\mfr[n]{i}{n_{p}}}{n} \nonumber \\
       &=& \summation_{i=1}^{n_{c}} \underbrace{\frc{\mfr[n]{i}{1}}{\mfr[n]{}{1}}}_{\red{\mfr[x]{i}{1}}} \cdot\underbrace{\frc{\mfr[n]{}{1}}{n}}_{\red{\mfr[\Pi]{}{1}}}\mfr[\mu]{i}{1} + \cdots + \summation_{i=1}^{n_{c}} \underbrace{\frc{\mfr[n]{i}{n_{p}}}{\mfr[n]{}{n_{p}}}}_{\red{\mfr[x]{i}{n_{p}}}} \cdot\underbrace{\frc{\mfr[n]{}{n_{p}}}{n}}_{\red{\mfr[\Pi]{}{n_{p}}}}\mfr[\mu]{i}{n_{p}} \nonumber \\
      &=& \summation_{i=1}^{n_{c}} \mfr[\Pi]{}{1}\mfr[x]{i}{1}\mfr[\mu]{i}{1} + \cdots + \summation_{i=1}^{n_{c}} \mfr[\Pi]{}{n_{p}}\mfr[x]{i}{n_{p}}\mfr[\mu]{i}{n_{p}} \nonumber \\
      &=& \summation_{j=1}^{n_{p}}\left[\summation_{i=1}^{n_{c}} \mfr[\mu]{}{j}\mfr[x]{i}{j}\mfr[\mu]{i}{j}\right].
\end{eqnarray}
Using the linear constraint (Eqn.~\ref{Chapter:ThermodynamicFormulation:Eqn:FractionConstraints2}),
\begin{shaded}\noindent
   \begin{equation}
      g = \summation_{j=1}^{n_{p}}\left[\summation_{i=1}^{n_{c}-1} \mfr[\Pi]{}{j}\mfr[x]{i}{j}\left(\mfr[\mu]{i}{j}-\mfr[\mu]{n_{c}}{j}\right) + \mfr[\Pi]{}{j}\mfr[\mu]{n_{c}}{j}\right]\label{Chapter:ThermodynamicFormulation:Eqn:GibbsMolar1}
   \end{equation}
\end{shaded}

From Eqn.~\ref{Chapter:ThermodynamicFormulation:Eqn:ChemPotDef2} and
\begin{displaymath}
   \mfr[\Pi]{}{k} = 1 - \summation_{j=1,j\ne k}{n_{p}} \mfr[\Pi]{}{j},
\end{displaymath}
for 2 phase systems $\left(\text{i.e., } \Pi = L, V\right)$, 
\begin{displaymath}
  \left \{
  \begin{aligned}
    & V = 1 - L && \\
    & V\mfr[x]{i}{V} = z_{i} - L\mfr[x]{i}{L},
  \end{aligned} \right.
\end{displaymath} 
and Eqn.~\ref{Chapter:ThermodynamicFormulation:Eqn:GibbsMolar1} becomes
\begin{equation}
  g = \summation_{i=1}^{n_{c}-1} L\mfr[x]{i}{L}\left[\left(\mfr[\mu]{i}{L}-\mfr[\mu]{i}{V}\right)-\left(\mfr[\mu]{n_{c}}{L}-\mfr[\mu]{n_{c}}{V}\right)\right] + L\left(\mfr[\mu]{n_{c}}{L}-\mfr[\mu]{n_{c}}{V}\right) +  \left(1 - \summation_{i=1}^{n_{c}-1}z_{i}\right)\mfr[\mu]{n_{c}}{V}
\end{equation}
However, as
\begin{displaymath}
   z_{n_{c}} = 1 - \summation_{i=1}^{n_{c}-1}z_{i},
\end{displaymath}
Thus,
\begin{shaded}\noindent
   \begin{equation}
  g = \summation_{i=1}^{n_{c}-1} L\mfr[x]{i}{L}\left[\left(\mfr[\mu]{i}{L}-\mfr[\mu]{i}{V}\right)-\left(\mfr[\mu]{n_{c}}{L}-\mfr[\mu]{n_{c}}{V}\right)\right] + L\left(\mfr[\mu]{n_{c}}{L}-\mfr[\mu]{n_{c}}{V}\right) + \summation_{i=1}^{n_{c}}z_{i}\mfr[\mu]{i}{V}.\label{Chapter:ThermodynamicFormulation:Eqn:GibbsMolarFinal}
   \end{equation}
\end{shaded}

The chemical potential may be expressed as a function of the fugacity, $f$,
\begin{equation}
   \mfr[\mu]{i}{j} = RT\left[\ln{\mfr[f]{i}{j}}-\ln{\left(P\mfr[x]{i}{j}\right)}\right] + \mfr[\theta]{i}{j}(T),
\end{equation}
with
\begin{displaymath}
  \left \{
  \begin{aligned}
    & 0 < \mfr[x]{i}{j} < 1 &&  \\
    & 0 < \mfr[\Pi]{}{j} < 1 && \text{ with }  i = 1, \cdots, n_{c},\;\; j = 1, \cdots, n_{p}
  \end{aligned} \right.
\end{displaymath} 

%%% Section
\section{Minimum Free Gibbs Energy}\label{Chapter:ThermodynamicFormulation:Section:MinimumGibbsEnergy}
In multiphase systems at constant pressure and temperature conditions, the concentration of all $n_{c}$ species in equilibrium are unknown. The minimum energy principle states that at equilibrium, the extensive parameters result in the minimum Gibbs free energy at constant pressure and temperature~\citep{Callen_Book}. Therefore,  

\begin{shaded}
   \begin{center}
     {\bf Statement of the VLE Problem}
   \end{center}

   Given $T$, $P$ and $z_{i}$ $\left(i=1,2,\cdots,n_{c}\right)$. Find the set of $\mfr[x]{1}{L},\cdots,\mfr[x]{n_{c}-1}{L}$ and $L$ that minimises
   \begin{displaymath}
      g\left(\mfr[x]{1}{L},\cdots,\mfr[x]{n_{c}-1}{L}, L\right) = \summation_{i=1}^{n_{c}-1}L\mfr[x]{i}{L}\left[\left(\mfr[\mu]{i}{L}-\mfr[\mu]{i}{V}\right)-\left(\mfr[\mu]{n_{c}}{L}-\mfr[\mu]{n_{c}}{V}\right)\right] + L\left(\mfr[\mu]{i}{L}-\mfr[\mu]{i}{V}\right) + \summation_{i}^{n_{c}}z_{i}\mfr[\mu]{i}{V} 
   \end{displaymath}
   with the constraints:
\[ 
\left\{
  \begin{tabular}{l}
  $0 < L < 1$ \\
  $0 < \mfr[x]{i}{L} < 1$, \hspace{1cm} $i=1,2,\cdots,n_{c}-1$\\
  $\summation_{i=1}^{n_{c}}\mfr[x]{i}{L} < 1$\\ 
  \end{tabular}
\right.
\]
where,
\begin{displaymath}
   \mfr[\mu]{i}{L}=\mfr[\mu]{i}{L}\left(\mfr[x]{1}{L},\cdots,\mfr[x]{n_{c}-1}{L}\right)\text{ and }\mfr[\mu]{i}{V}=\mfr[\mu]{i}{V}\left(\mfr[x]{1}{L},\cdots,\mfr[x]{n_{c}-1}{L}, L\right)
\end{displaymath}

\end{shaded}
%\end{comment}
