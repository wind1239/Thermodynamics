
%%%
%%% CHAPTER
%%%
\chapter{Hydrate Formation}\label{Chapter:Hydrate}


%%% Section
\section{Mass Conservation and Degrees of Freedom Analysis}\label{Chapter:Hydrate:Section:MassConservation}

Assuming an isothermal flash with the following phases present in the equilibrium:
  \begin{center}
    \begin{tabular}{l l l l}
      $L$: & liquid phase & $H$: & hydrate phase \\
      $V$: & vapour phase & $I$: & ice phase \\ 
    \end{tabular}
  \end{center}
$L$, $V$, $H$ and $I$ are the molar fraction of each phase. In addition, $F$ represents the molar fraction of the feeding stream. Finally, let's define $x_{i}^{j}$ as the molar fraction of component $i \left(=1,2,\cdots,n_{c}\right)$ in phase $j \left(=L,V,H,I,F\right)$. Thus, assuming a closed system (\ie no mass transfer across a defined domain border):
  \begin{eqnarray}
     && L + V + H + I = 1 \\\label{Chapter:Hydrate:Eqn:Balance:MassFractionPhase}
     && \mfr[x]{i}{F}F = \mfr[x]{i}{V}V + \mfr[x]{i}{L}L + \mfr[x]{i}{H}H + \mfr[x]{i}{I}I, \label{Chapter:Hydrate:Eqn:Balance:MassFractionComponent}
  \end{eqnarray}
with the following linear constraints:
  \begin{equation}
    \summation_{i}^{n_{c}} \mfr[x]{i}{j} = 1, \hspace{2cm} j=L,V,H,I,F\label{Chapter:Hydrate:Eqn:Balance:MassFractionConstraint}
  \end{equation}
 
The resulting system of non-linear algebraic equations have $4n_{c}+6$ unknowns, \ie $\mfr[x]{i}{V}, \mfr[x]{i}{L}, \mfr[x]{i}{H}, \mfr[x]{i}{I}$ (each of them with $n_{c}$ components), $L$, $V$, $I$, $H$, temperature ($T$) and pressure ($P$).  In the case of methane hydrate, $n_{c}=2$ $\left(\text{\ie} CH_{4}+H_{2}O\right)$. Additionally, the ice phase ($I$) contains just water -- $w_{w}^{I}=1$, and the number of unknowns and equations (linear constraint for ice) are reduced.

In the equilibrium, the equality of phase fugacities for a given component $i$,
   \begin{equation}
      \mfr[f]{i}{V} = \mfr[f]{i}{L} = \mfr[f]{i}{H} = \mfr[f]{i}{I} \hspace{2cm}\text{with } i=1,\cdots,n_{c} \label{Chapter:Hydrate:Eqn:FugacityEquality}
   \end{equation}
Note that one of the relations in Eqn.~\ref{Chapter:Hydrate:Eqn:FugacityEquality} is not independent. Thus the total number of independent equations is $4n_{c}+4$, and the number of degrees of freedom $\left(\varphi\right)$, \ie   
   \begin{center} 
      (number of unknowns) - (number of independent equations),
   \end{center}
is equal to $2$.

If we specify the pair $\left[T,P\right]$, we can determine the composition of each component, $\mfr[x]{i}{j}$, and each phase $j$. The Gibb's phase rule,
  \begin{eqnarray}
     \varphi &=& 2 + n_{c} - n_{p} \\ \nonumber
             &=& 2 + 2 - n_{p} = 4 - n_{p},
  \end{eqnarray}
states the possibility of a quadruple point $\left(\varphi=0\right)$, however if we want to specify $\left[T,P\right]$ then a two-phase system may occur.
