%\documentclass[11pts,a4paper,amsmath,amssymb,floatfix]{article}%{report}%{book}
\documentclass[12pts,a4paper,amsmath,amssymb,floatfix]{article}%{report}%{book}
\usepackage{graphicx,wrapfig}% Include figure files
%\usepackage{dcolumn,enumerate}% Align table columns on decimal point
\usepackage{bm,dpfloat}% bold math
\usepackage[pdftex,bookmarks,colorlinks=true,urlcolor=rltblue,citecolor=blue]{hyperref}
\usepackage{amsfonts,amsmath,amssymb,stmaryrd,indentfirst}
%\usepackage{chemist}
\usepackage{times,psfrag}
\usepackage{natbib} 
\usepackage{color}
\usepackage{units}
\usepackage{rotating}
\usepackage{multirow}
\usepackage{gantt}
\usepackage{pdflscape}
\usepackage{comment}
 
\usepackage{enumerate}%,enumitem}% Align table columns on decimal point

% Text + Eqn box:
\usepackage[framemethod=TikZ]{mdframed}
\usepackage{lipsum}
\mdfdefinestyle{JFrame}{%
    linecolor=blue,
    outerlinewidth=1pt,
    roundcorner=20pt,
    innertopmargin=\baselineskip,
    innerbottommargin=\baselineskip,
    innerrightmargin=20pt,
    innerleftmargin=20pt,
    backgroundcolor=blue!10!white}

%\usepackage{pifont}
%\usepackage{subfigure}
%\usepackage{subeqnarray}
%\usepackage{ifthen}
 
\usepackage{supertabular}
\usepackage{moreverb}
\usepackage{listings} 
\usepackage{palatino}
%\usepackage{doi}
\usepackage{longtable}
\usepackage{float}
\usepackage{perpage}
\MakeSorted{figure}
%\usepackage{pdflscape}


\definecolor{rltblue}{rgb}{0,0,0.75}


%\usepackage{natbib}
\usepackage{fancyhdr} %%%%
\pagestyle{fancy}%%%%
% with this we ensure that the chapter and section
% headings are in lowercase
%%%%\renewcommand{\chaptermark}[1]{\markboth{#1}{}}
\renewcommand{\sectionmark}[1]{\markright{\thesection\ #1}}
\fancyhf{} %delete the current section for header and footer
\fancyhead[LE,RO]{\bfseries\thepage}
\fancyhead[LO]{\bfseries\rightmark}
\fancyhead[RE]{\bfseries\leftmark}
\renewcommand{\headrulewidth}{0.5pt}
% make space for the rule
\fancypagestyle{plain}{%
\fancyhead{} %get rid of the headers on plain pages
\renewcommand{\headrulewidth}{0pt} % and the line
}

\def\newblock{\hskip .11em plus .33em minus .07em}
\usepackage{color}

%\usepackage{makeidx}
%\makeindex

\setlength\textwidth      {16.cm}
\setlength\textheight     {21.6cm}
\setlength\oddsidemargin  {-0.3cm}
\setlength\evensidemargin {0.3cm}

\setlength\headheight{14.49998pt} 
\setlength\topmargin{0.0cm}
\setlength\headsep{1.cm}
\setlength\footskip{1.cm}
\setlength\parskip{0pt}
\setlength\parindent{0pt}


%%%
%%% Headers and Footers
\lhead[] {\text{\small{CompRS}}} 
\rhead[] {{\text{\small{CoMaELa}}}}
%\chead[] {\text{\small{Session 2012/13}}} 
\lfoot[]{{\today}}
%\cfoot[\thepage]{\thepage}
\rfoot[\text{\small{\thepage}}]{\thepage}
\renewcommand{\headrulewidth}{0.8pt}



\usepackage[T1]{fontenc}
\usepackage[utf8]{inputenc}
\usepackage{lmodern}
\usepackage[version=3]{mhchem}


\makeatletter
\newcounter{reaction}
%%% >> for article <<
\renewcommand\thereaction{C\,\arabic{reaction}}
%%% << for article <<
%%% >> for report and book >>
%\renewcommand\thereaction{C\,\thechapter.\arabic{reaction}}
%\@addtoreset{reaction}{chapter}
%%% << for report and book <<
\newcommand\reactiontag{\refstepcounter{reaction}\tag{\thereaction}}
\newcommand\reaction@[2][]{\begin{equation}\ce{#2}%
\ifx\@empty#1\@empty\else\label{#1}\fi%
\reactiontag\end{equation}}
\newcommand\reaction@nonumber[1]{\begin{equation*}\ce{#1}%
\end{equation*}}
\newcommand\reaction{\@ifstar{\reaction@nonumber}{\reaction@}}
\makeatother


%%%
%%% Notes
%%%
\newcommand{\JGnote}[1]{\fbox{\parbox{\textwidth}{\color{blue} JG: #1}}}



%%%
%%% space between lines
%%%
\renewcommand{\baselinestretch}{1.5}

\newenvironment{VarDescription}[1]%
  {\begin{list}{}{\renewcommand{\makelabel}[1]{\textbf{##1:}\hfil}%
    \settowidth{\labelwidth}{\textbf{#1:}}%
    \setlength{\leftmargin}{\labelwidth}\addtolength{\leftmargin}{\labelsep}}}%
  {\end{list}}

%%%%%%%%%%%%%%%%%%%%%%%%%%%%%%%%%%%%%%%%%%%
%%%%%%                              %%%%%%%
%%%%%%      NOTATION SECTION        %%%%%%%
%%%%%%                              %%%%%%%
%%%%%%%%%%%%%%%%%%%%%%%%%%%%%%%%%%%%%%%%%%%

% Text abbreviations.
\newcommand{\ie}{{\em{i.e., }}}
\newcommand{\eg}{{\em{e.g., }}}
%XS\newcommand{\cf}{{\em{cf., }}}
\newcommand{\wrt}{with respect to}
\newcommand{\lhs}{left hand side}
\newcommand{\rhs}{right hand side}
% Commands definining mathematical notation.

% This is for quantities which are physically vectors.
\renewcommand{\vec}[1]{{\mbox{\boldmath$#1$}}}
% Physical rank 2 tensors
\newcommand{\tensor}[1]{\overline{\overline{#1}}}
% This is for vectors formed of the value of a quantity at each node.
\newcommand{\dvec}[1]{\underline{#1}}
% This is for matrices in the discrete system.
\newcommand{\mat}[1]{\mathrm{#1}}


\DeclareMathOperator{\sgn}{sgn}
\newtheorem{thm}{Theorem}[section]
\newtheorem{lemma}[thm]{Lemma}

%\newcommand\qed{\hfill\mbox{$\Box$}}
\newcommand{\re}{{\mathrm{I}\hspace{-0.2em}\mathrm{R}}}
\newcommand{\inner}[2]{\langle#1,#2\rangle}
\renewcommand\leq{\leqslant}
\renewcommand\geq{\geqslant}
\renewcommand\le{\leqslant}
\renewcommand\ge{\geqslant}
\renewcommand\epsilon{\varepsilon}
\newcommand\eps{\varepsilon}
\renewcommand\phi{\varphi}
\newcommand{\bmF}{\vec{F}}
\newcommand{\bmphi}{\vec{\phi}}
\newcommand{\bmn}{\vec{n}}
\newcommand{\bmns}{{\textrm{\scriptsize{\boldmath $n$}}}}
\newcommand{\bmi}{\vec{i}}
\newcommand{\bmj}{\vec{j}}
\newcommand{\bmk}{\vec{k}}
\newcommand{\bmx}{\vec{x}}
\newcommand{\bmu}{\vec{u}}
\newcommand{\bmv}{\vec{v}}
\newcommand{\bmr}{\vec{r}}
\newcommand{\bma}{\vec{a}}
\newcommand{\bmg}{\vec{g}}
\newcommand{\bmU}{\vec{U}}
\newcommand{\bmI}{\vec{I}}
\newcommand{\bmq}{\vec{q}}
\newcommand{\bmT}{\vec{T}}
\newcommand{\bmM}{\vec{M}}
\newcommand{\bmtau}{\vec{\tau}}
\newcommand{\bmOmega}{\vec{\Omega}}
\newcommand{\pp}{\partial}
\newcommand{\kaptens}{\tensor{\kappa}}
\newcommand{\tautens}{\tensor{\tau}}
\newcommand{\sigtens}{\tensor{\sigma}}
\newcommand{\etens}{\tensor{\dot\epsilon}}
\newcommand{\ktens}{\tensor{k}}
\newcommand{\half}{{\textstyle \frac{1}{2}}}
\newcommand{\tote}{E}
\newcommand{\inte}{e}
\newcommand{\strt}{\dot\epsilon}
\newcommand{\modu}{|\bmu|}
% Derivatives
\renewcommand{\d}{\mathrm{d}}
\newcommand{\D}{\mathrm{D}}
\newcommand{\ddx}[2][x]{\frac{\d#2}{\d#1}}
\newcommand{\ddxx}[2][x]{\frac{\d^2#2}{\d#1^2}}
\newcommand{\ddt}[2][t]{\frac{\d#2}{\d#1}}
\newcommand{\ddtt}[2][t]{\frac{\d^2#2}{\d#1^2}}
\newcommand{\ppx}[2][x]{\frac{\partial#2}{\partial#1}}
\newcommand{\ppxx}[2][x]{\frac{\partial^2#2}{\partial#1^2}}
\newcommand{\ppt}[2][t]{\frac{\partial#2}{\partial#1}}
\newcommand{\pptt}[2][t]{\frac{\partial^2#2}{\partial#1^2}}
\newcommand{\DDx}[2][x]{\frac{\D#2}{\D#1}}
\newcommand{\DDxx}[2][x]{\frac{\D^2#2}{\D#1^2}}
\newcommand{\DDt}[2][t]{\frac{\D#2}{\D#1}}
\newcommand{\DDtt}[2][t]{\frac{\D^2#2}{\D#1^2}}
% Norms
\newcommand{\Ltwo}{\ensuremath{L_2} }
% Basis functions
\newcommand{\Qone}{\ensuremath{Q_1} }
\newcommand{\Qtwo}{\ensuremath{Q_2} }
\newcommand{\Qthree}{\ensuremath{Q_3} }
\newcommand{\QN}{\ensuremath{Q_N} }
\newcommand{\Pzero}{\ensuremath{P_0} }
\newcommand{\Pone}{\ensuremath{P_1} }
\newcommand{\Ptwo}{\ensuremath{P_2} }
\newcommand{\Pthree}{\ensuremath{P_3} }
\newcommand{\PN}{\ensuremath{P_N} }
\newcommand{\Poo}{\ensuremath{P_1P_1} }
\newcommand{\PoDGPt}{\ensuremath{P_{-1}P_2} }

\newcommand{\metric}{\tensor{M}}
\newcommand{\configureflag}[1]{\texttt{#1}}

% Units
\newcommand{\m}[1][]{\unit[#1]{m}}
\newcommand{\km}[1][]{\unit[#1]{km}}
\newcommand{\s}[1][]{\unit[#1]{s}}
\newcommand{\invs}[1][]{\unit[#1]{s}\ensuremath{^{-1}}}
\newcommand{\ms}[1][]{\unit[#1]{m\ensuremath{\,}s\ensuremath{^{-1}}}}
\newcommand{\mss}[1][]{\unit[#1]{m\ensuremath{\,}s\ensuremath{^{-2}}}}
\newcommand{\K}[1][]{\unit[#1]{K}}
\newcommand{\PSU}[1][]{\unit[#1]{PSU}}
\newcommand{\Pa}[1][]{\unit[#1]{Pa}}
\newcommand{\kg}[1][]{\unit[#1]{kg}}
\newcommand{\rads}[1][]{\unit[#1]{rad\ensuremath{\,}s\ensuremath{^{-1}}}}
\newcommand{\kgmm}[1][]{\unit[#1]{kg\ensuremath{\,}m\ensuremath{^{-2}}}}
\newcommand{\kgmmm}[1][]{\unit[#1]{kg\ensuremath{\,}m\ensuremath{^{-3}}}}
\newcommand{\Nmm}[1][]{\unit[#1]{N\ensuremath{\,}m\ensuremath{^{-2}}}}

% Dimensionless numbers
\newcommand{\dimensionless}[1]{\mathrm{#1}}
\renewcommand{\Re}{\dimensionless{Re}}
\newcommand{\Ro}{\dimensionless{Ro}}
\newcommand{\Fr}{\dimensionless{Fr}}
\newcommand{\Bu}{\dimensionless{Bu}}
\newcommand{\Ri}{\dimensionless{Ri}}
\renewcommand{\Pr}{\dimensionless{Pr}}
\newcommand{\Pe}{\dimensionless{Pe}}
\newcommand{\Ek}{\dimensionless{Ek}}
\newcommand{\Gr}{\dimensionless{Gr}}
\newcommand{\Ra}{\dimensionless{Ra}}
\newcommand{\Sh}{\dimensionless{Sh}}
\newcommand{\Sc}{\dimensionless{Sc}}


% Journals
\newcommand{\IJHMT}{{\it International Journal of Heat and Mass Transfer}}
\newcommand{\NED}{{\it Nuclear Engineering and Design}}
\newcommand{\ICHMT}{{\it International Communications in Heat and Mass Transfer}}
\newcommand{\NET}{{\it Nuclear Engineering and Technology}}
\newcommand{\HT}{{\it Heat Transfer}}   
\newcommand{\IJHT}{{\it International Journal for Heat Transfer}}

\newcommand{\frc}{\displaystyle\frac}
\newenvironment{frcseries}{\fontfamily{frc}\selectfont}{}
\newcommand{\textfrc}[1]{{\frcseries#1}}
\newcommand{\summation}{\sum\limits}
\newcommand{\red}{\textcolor{red}} 
\newcommand{\blue}{\textcolor{blue}} 
\newcommand{\green}{\textcolor{green}} 


%\usepackage{enumitem}%
%\newlist{ExList}{enumerate}{1}
%\setlist[ExList,1]{label={\bf Example 1.} {\bf \arabic*}}

%\newlist{ProbList}{enumerate}{1}
%\setlist[ProbList,1]{label={\bf Problem 1.} {\bf \arabic*}}

%%%%%%%%%%%%%%%%%%%%%%%%%%%%%%%%%%%%%%%%%%%
%%%%%%                              %%%%%%%
%%%%%% END OF THE NOTATION SECTION  %%%%%%%
%%%%%%                              %%%%%%%
%%%%%%%%%%%%%%%%%%%%%%%%%%%%%%%%%%%%%%%%%%%


% Cause numbering of subsubsections. 
%\setcounter{secnumdepth}{8}
%\setcounter{tocdepth}{8}

\setcounter{secnumdepth}{4}%
\setcounter{tocdepth}{4}%

\begin{document}


\begin{center}
{\Large {\bf Developing an Integrated EOS-based Compositional Simulator Modelling Framework for Environmental and Industrial Flow Applications}}
\end{center}
\begin{flushright}
\today
\end{flushright}

%%%
%%% SECTION
%%%
\section{Thermodynamic Formulation}
 
%%% SUB-SECTION
\subsection{Mass Balance}
For a system with $n_{p}$ phases and $n_{c}$ components, the mass of each phase $j$ is,
\begin{equation}
m^{j} = \summation_{i=1}^{n_{c}}m_{i}^{j}\;\;\;\text{ for }j=1,2,\cdots,n_{p}
\label{Eqn_MassBalance_Phase}
\end{equation}

And the mass of each component $i\left(=1,2,\cdots,n_{c}\right)$,
\begin{equation}
m_{i} = m_{i}^{1} + m_{i}^{2} + \cdots + m_{i}^{n_{p}} = \summation_{j=1}^{n_{p}}m_{i}^{j}
\label{Eqn_MassBalance_Component}
\end{equation}

The total mass $m$ can be define as,
\begin{equation}
m - m^{1} + m^{2} + \cdots + m^{n_{p}} = \summation_{j=1}^{n_{p}}m^{j} = \summation_{i=1}^{n_{c}}m_{i}
\label{Eqn_MassBalance_Mass}
\end{equation}

Defining the intensive properties:
\begin{eqnarray}
w_{i}^{j} = \frc{m_{i}^{j}}{m^{j}} \label{Eqn_MassBalance_MassFraction} \\
z_{i} = \frc{m_{i}}{m} \label{Eqn_MassBalance_MassFractionFeeding} \\
\Pi^{j} = \frc{m^{j}}{m}\label{Eqn_MassBalance_PhaseMassFraction} 
\end{eqnarray}
where $w_{i}^{j}$ is the mass fraction of component $i$ in phase $j$, $\Pi^{j}$ is the mass fraction of phase $j$ and $z_{i}$ is the overall feed mass fraction of component $i$. From Eqns.~\ref{Eqn_MassBalance_Mass}-~\ref{Eqn_MassBalance_PhaseMassFraction}, we can define normalised quantities, i.e., fractions, and we can add the following constraints:
\begin{eqnarray}
&&w_{n_{c}}^{j} = 1 - \summation_{i=1}^{n_{c}-1}w_{i}^{j}  \hspace{3cm} j=1,2,\cdots,n_{p}  \label{Eqn_MassBalance_MassFraction2} \\
&&z_{n_{c}} = 1 - \summation_{i=1}^{n_{c}-1}z_{1}  \\
&&\Pi^{n_{p}} = 1 - \left(\Pi^{1} + \Pi^{2} + \cdots + \Pi^{n_{p}-1}\right) = 1 - \summation_{j=1}^{n_{p}-1} \Pi^{j}\label{Eqn_MassBalance_PhaseMassFraction2}
\end{eqnarray} 
with
\begin{displaymath}
  z_{i} = \frc{m_{i}}{m} = \frc{\summation_{j=1}^{n_{p}}m_{i}^{j}}{m} \frc{m^{j}}{m^{j}} = \underbrace{\frc{m^{j}}{m}}_{\red{\Pi^{j}}} \cdot \overbrace{\frc{\summation_{j=1}^{n_{p}}m_{i}^{j}}{m^{j}}}^{\red{\summation_{j=1}^{n_{p}}w_{i}^{j}}} 
\end{displaymath}
leading to
\begin{equation}
z_{i} = \Pi^{1}w_{i}^{1} + \Pi^{2}w_{i}^{2} + \cdots + \Pi^{n_{p}}w_{i}^{n_{p}}
\label{Eqn_MassBalance_FeedMassFractionConstraint}
\end{equation}
with
\begin{equation}
0\leq\Pi^{j}\leq 1 \hspace{3cm} 0\leq w_{i}^{j}\leq 1
\end{equation}

If the solution is contained within \red{$n_{p}$} phases, the inequality can be rewritten as,
\begin{equation}
0 < \Pi^{j} < 1
\end{equation}
Thus
\begin{displaymath}
\Pi^{k} = 1 - \summation_{j=1,j\neq k}^{n_{p}} \Pi^{j} \neq 0
\end{displaymath}
Therefore from Eqn.~\ref{Eqn_MassBalance_FeedMassFractionConstraint},
\begin{equation}
w_{i}^{k} = \frc{z_{i}+\summation_{j=1,j\neq k}^{n_{p}}\Pi^{j}w_{i}^{j}}{\Pi^{k}} = \frc{z_{i} + \summation_{j=1,j\neq k}^{n_{p}}\Pi^{j}w_{i}^{j}}{1-\summation_{j=1,j\neq k}^{n_{p}}\Pi^{j}}
\label{Eqn_MassBalance_FeedMassFractionConstraint2}
\end{equation}


\begin{mdframed}[style=JFrame]
For 2 phases, $\Pi^{1}=L$ and $\Pi^{2}=V$, Eqn.~\ref{Eqn_MassBalance_FeedMassFractionConstraint} becomes:
\begin{displaymath}
w_{i}^{V} = \frc{z_{i}-Lw_{i}^{L}}{1-L}
\end{displaymath}
for $i=1,2,\cdots,n_{c}$, and for 3 phases,$\Pi^{1}=L$, $\Pi^{2}=V$ and $\Pi^{3}=H$:
\begin{displaymath}
w_{i}^{H} = \frc{z_{i}-\left(V w_{i}^{V} + L w_{i}^{L}\right)}{1 - \left(V w_{i}^{V} + L w_{i}^{L}\right)}
\end{displaymath}
\end{mdframed}

%% SUB-SECTION
\subsection{Free Gibbs Energy}
For equilibrium problems (e.g., vapour-liquid equilibrium, VLE), the total free Gibbs energy can be expressed as,
\begin{equation}
G = \summation_{j=1}^{n_{p}}\left(\summation_{i=1}^{n_{c}} m_{i}^{j}\mu_{i}^{j}\right)
\label{MassGibbs_Definition}
\end{equation} 
this represents
\begin{displaymath}
G = \summation_{i=1}^{n_{c}} m_{i}^{1}\mu_{i}^{1} + \summation_{i=1}^{n_{c}} m_{i}^{2}\mu_{i}^{2} + \cdots + \summation_{i=1}^{n_{c}} m_{i}^{n_{p}}\mu_{i}^{n_{p}}
\end{displaymath}
The chemical potential $\left(\mu\right)$ can be written as a function of the mass of each component at each phase,
\begin{eqnarray}
\mu_{i}^{j} &=& \mu_{i}^{j}\left(m_{1}^{j}, m_{2}^{j}, \cdots, m_{n_{c}}^{j}\right) = \mu_{i}^{j}\left(\frc{m_{1}^{j}}{m^{j}}, \frc{m_{2}^{j}}{m^{j}}, \cdots, \frc{m_{n_{c}}^{j}}{m^{j}}\right)\nonumber \\
 &=& \mu_{i}^{j}\left(w_{1}^{j}, w_{2}^{j}, \cdots, w_{n_{c}}^{j}\right)\label{chempotential_functional}
\end{eqnarray}
Dividing Eqn.~\ref{MassGibbs_Definition} by $m$,
\begin{equation}
g = \frc{G}{m} = \summation_{j=1}^{n_{p}}\left(\summation_{i=1}^{n_{c}}\frc{m_{i}^{j}}{m}\mu_{i}^{j}\right) = \summation_{i=1}^{n_{c}}\frc{m_{i}^{1}}{m^{1}} \cdot \frc{m^{1}}{m} \mu_{i}^{1} + \cdots + \summation_{i=1}^{n_{c}}\frc{m_{i}^{n_{p}}}{m^{n_{p}}} \cdot \frc{m^{n_{p}}}{m} \mu_{i}^{n_{p}}
\label{MassGibbs_Definition2}
\end{equation}
And we rewrite Eqn.~\ref{MassGibbs_Definition2} as,
\begin{equation}
g = \summation_{i=1}^{n_{c}}\Pi^{1}w_{i}^{1}\mu_{i}^{1} + \summation_{i=1}^{n_{c}}\Pi^{2}w_{i}^{2}\mu_{i}^{2} + \cdots + \summation_{i=1}^{n_{c}}\Pi^{n_{p}}w_{i}^{n_{p}}\mu_{i}^{n_{p}} = \summation_{j=1}^{n_{p}}\left[\summation_{i}^{n_{c}}\Pi^{j}w_{i}^{j}\mu_{i}^{j}\right] \label{MassGibbs_Definition3}
\end{equation}


\begin{mdframed}[style=JFrame]
For 2 ($L$ and $V$) phases,
\begin{displaymath}
g = \summation_{i=1}^{n_{c}}Lw_{i}^{L}\mu_{i}^{L} + \summation_{i=1}^{n_{c}}Vw_{i}^{V}\mu_{i}^{V}
\end{displaymath}
and 3 ($L$, $V$ and $H$) phases,
\begin{displaymath}
g = \summation_{i=1}^{n_{c}}Lw_{i}^{L}\mu_{i}^{L} + \summation_{i=1}^{n_{c}}Vw_{i}^{V}\mu_{i}^{V} + \summation_{i=1}^{n_{c}}Hw_{i}^{H}\mu_{i}^{H}
\end{displaymath}
\end{mdframed}


Now replacing Eqn.~\ref{Eqn_MassBalance_MassFraction2} in Eqn.~\ref{MassGibbs_Definition3},
\begin{equation}
g = \summation_{i=1}^{n_{c}-1}\Pi^{1}w_{i}^{1}\left(\mu_{i}^{1}-\mu_{n_{c}}^{1}\right) + \Pi^{1}\mu_{n_{c}}^{1} + \summation_{i=1}^{n_{c}-1}\Pi^{2}w_{i}^{2}\left(\mu_{i}^{2}-\mu_{n_{c}}^{2}\right) + \Pi^{2}\mu_{n_{c}}^{2} + \cdots + \summation_{i=1}^{n_{c}-1}\Pi^{n_{p}}w_{i}^{n_{p}}\left(\mu_{i}^{n_{p}}-\mu_{n_{c}}^{n_{p}}\right) + \Pi^{n_{p}}\mu_{n_{c}}^{n_{p}}
\label{MassGibbs_Definition4}
\end{equation}
Now replacing Eqns.~\ref{Eqn_MassBalance_PhaseMassFraction2} and~\ref{Eqn_MassBalance_FeedMassFractionConstraint2} in~\ref{MassGibbs_Definition4}:
\begin{eqnarray}
g &=& \summation_{i=1}^{n_{c}}\Pi^{1}w_{i}^{1}\left(\mu_{i}^{1}-\mu_{n_{c}}^{1}\right) + \Pi^{1}\mu_{n_{c}}^{1} + \cdots + \summation_{i=1}^{n_{c}}\Pi^{n_{p}-1}w_{i}^{n_{p}-1}\left(\mu_{i}^{n_{p}-1}-\mu_{n_{c}}^{n_{p}-1}\right) + \Pi^{n_{p}-1}\mu_{n_{c}}^{n_{p}-1} + \nonumber \\
   && \summation_{i=1}^{n_{c}-1}\left[z_{i} - \summation_{j=1}^{n_{p}-1}\Pi^{j}w_{i}^{j}\right]\left(\mu_{i}^{n_{p}}-\mu_{n_{c}}^{n_{p}}\right) + \Pi^{n_{p}}\mu_{n_{c}}^{n_{p}}\nonumber
\end{eqnarray}

Rearranging
\begin{eqnarray}
g &=& \summation_{i=1}^{n_{c}}\Pi^{1}w_{i}^{1}\left[\left(\mu_{i}^{1}-\mu_{i}^{n_{p}}\right) - \left(\mu_{n_{c}}^{1}-\mu_{n_{c}}^{n_{p}}\right)\right] + \cdots + \summation_{i=1}^{n_{c}}\Pi^{n_{p}-1}w_{i}^{n_{p}-1}\left[\left(\mu_{i}^{n_{p}-1}-\mu_{i}^{n_{p}}\right) - \left(\mu_{n_{c}}^{n_{p}-1}-\mu_{n_{c}}^{n_{p}}\right)\right] + \nonumber \\
&& \summation_{j=1}^{n_{p}-1}\Pi^{j}\mu_{n_{c}}^{j} + \summation_{i=1}^{n_{c}-1}z_{i}\left(\mu_{i}^{n_{p}}-\mu_{n_{c}}^{n_{p}}\right) +\left(1-\summation_{j=1}^{n_{p}-1}\Pi^{j}\right)\mu_{n_{c}}^{n_{p}} \nonumber
\end{eqnarray}

With a compressed notation:

\begin{eqnarray}
g &=& \summation_{i=1}^{n_{c}-1}\left\{\summation_{j=1}^{n_{p}-1}\Pi^{j}w_{i}^{j}\left[\left(\mu_{i}^{j}-\mu_{i}^{n_{p}}\right) - \left(\mu_{n_{c}}^{j}-\mu_{n_{c}}^{n_{p}}\right)\right]\right\} + \summation_{j=1}^{n_{p}-1}\Pi^{j}\left(\mu_{n_{c}}^{j}-\mu_{n_{c}}^{n_{p}}\right) +  \nonumber \\
&& \hspace{3cm} \summation_{i=1}^{n_{c}-1} z_{i}\mu_{i}^{n_{p}} + \underbrace{\left(1-\summation_{i=1}^{n_{c}-1}z_{i}\right)}_{\red{z_{n_{c}}}}\mu_{n_{c}}^{n_{p}} \nonumber 
\end{eqnarray}
Thus
\red{\begin{equation}
g = \summation_{i=1}^{n_{c}-1}\left\{\summation_{j=1}^{n_{p}-1}\Pi^{j}w_{i}^{j}\left[\left(\mu_{i}^{j}-\mu_{i}^{n_{p}}\right) - \left(\mu_{n_{c}}^{j}-\mu_{n_{c}}^{n_{p}}\right)\right]\right\} + \summation_{j=1}^{n_{p}-1}\Pi^{j}\left(\mu_{n_{c}}^{j}-\mu_{n_{c}}^{n_{p}}\right) + \summation_{i=1}^{n_{c}}z_{i}\mu_{i}^{n_{p}}
\end{equation}}

The chemical potential was defined as a functional (Eqn.~\ref{chempotential_functional}):
\begin{eqnarray}
&& \mu_{i}^{1} = \mu_{i}^{1}\left(w_{1}^{1},w_{2}^{1},\cdots,w_{n_{c}}^{1}\right) \nonumber \\
&& \mu_{i}^{2} = \mu_{i}^{2}\left(w_{1}^{2},w_{2}^{2},\cdots,w_{n_{c}}^{2}\right) \nonumber \\
&& \hspace{2cm}\vdots \nonumber \\
&& \mu_{i}^{n_{p}} = \mu_{i}^{n_{p}}\left(w_{1}^{n_{p}},w_{2}^{n_{p}},\cdots,w_{n_{c}}^{n_{p}}\right) \label{ChemPotDef2}
\end{eqnarray}
with constraint Eqn.~\ref{Eqn_MassBalance_FeedMassFractionConstraint2} in Eqn.~\ref{ChemPotDef2},
\begin{eqnarray}
\mu_{l}^{k} &=& \mu_{l}^{k}\left(w_{i}^{j},\Pi^{j}\right) \nonumber \\
           &=& \mu_{l}^{k}\left(w_{1}^{1},\cdots,w_{n_{c}-1}^{1},w_{1}^{2},\cdots,w_{n_{c}-1}^{2},\cdots,w_{1}^{n_{p}},w_{2}^{n_{p}},\cdots,w_{n_{c}-1}^{n_{p}},\Pi^{1},\cdots,\Pi^{n_{p}}\right)\nonumber
\end{eqnarray}
with $i = 1, 2, \cdots, l, \cdots, n_{c} - 1$, $j = 1, 2, \cdots, n_{p}$ and $j\neq k$

And the mass-based free Gibbs energy should take the following format
\begin{displaymath}
g = g\left(w_{i}^{j},\Pi^{j}\right).
\end{displaymath}
From classical thermodynamic, $\mu_{i}^{j}$ can be defined as a function of the fugacity coefficient, $\varphi_{i}^{j}$,
\begin{equation}
\mu_{i}^{j}= R T\left[\ln\varphi_{i}^{j} + \ln\left(P x_{i}^{j}\right)\right] + \theta_{i}\left(T\right),
\end{equation}
where $\theta_{i}$ is an integration constant. The fugacity coefficient is defined as,
\begin{equation}
\varphi_{i}^{j} = \frc{f_{i}^{j}}{P x_{i}^{j}},
\end{equation}
where $f$ and $x$ are the fugacity and molar fraction respectively.
\bibliographystyle{unsrt}
%\bibliographystyle{plainnat}
%\bibliographystyle{acm}
\bibliography{references}



\end{document}
